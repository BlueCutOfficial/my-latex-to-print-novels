% The text in chapter will appear on the chapter page, but this one is too long to be in the table of contents, so I use "*" to remove it.
\chapter*{Vous aussi, vous maîtrisez la magie~!}
% Then I manually add my own entry in the table of contents for the chapter, but I rename it to something shorter. 
\addcontentsline{toc}{chapter}{Merci !}

% textit is a command for font style italic 
\textit{Merci à toi, lecteur, d'avoir pris un moment pour voyager avec Suzuha. J'espère que cette aventure t'a plu et que tu repasseras un de ces jours par l'Académie des Renards, car elle a fort besoin de toi.} 

\parbr

L'Académie des Renards est une petite collection de romans indépendante. Vous seul avez le pouvoir de la faire connaître en la partageant autour de vous et en la suivant sur les réseaux sociaux. Les Renards comptent sur votre soutien~!

\parbr

% href command to create links in the PDF is provided by hyperref package. I use the same file for e-books and paper book, that's why links are implemented. In "papier.tex", hyperref is configured so no style appears on links and screw up the paper render.
Site web~: \href{http://academie-des-renards.dunstetter.fr}{academie-des-renards.dunstetter.fr} \\

% The \\ at the end of the line is a line breaker that add a little space before the next line. 
Facebook~: \href{https://www.facebook.com/academiedesrenards/}{www.facebook.com/academiedesrenards/} \\

Twitter~: \href{https://twitter.com/academierenards}{@academierenards} \\

Instagram~:  \href{https://www.instagram.com/academiedesrenards/}{@academiedesrenards} \\

Boutique~: \href{https://utip.io/academiedesrenards/shop}{https://utip.io/academiedesrenards/shop}

% After the last links, I jump to a new page for the last acknowledgements, for esthetic purpose.
\newpage

% textbf is a command for font style bold 
\textbf{Quelques mentions spéciales}

\parbr

Merci Guillaume pour ton soutien sans faille et, d'un point de vue plus pragmatique, pour ton coup de main sur la partie technique. J'aurais sans aucun doute ramé bien plus longtemps pour faire aboutir ce projet si tu n'avais pas été là pour me décharger sur la mise en forme. Et j'apprécie vraiment ce week-end entier passé à me relire, toi qui ne lis pourtant jamais. \\

Merci m'man, tu es sans conteste mon filet à fautes le plus efficace, et tes retours cruels et sans pitié m'ont permis d'améliorer mon écrit (mais non, je plaisante, ce n'était pas si méchant). \\

Merci J-S pour la relecture, j'ai trouvé ça vraiment sympa que tu dévores ce livre aussi vite après l'avoir commencé.



