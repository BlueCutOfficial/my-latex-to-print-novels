% We rely on the document class "book"
% I set the font size to 12pt and the ratio to A5, it's up to you
% oneside is the option that prevents the addition of a white page between two chapters if the first one ends on an odd page. In novels, we currently see chapters starting on left or right page, it doesn't matter. 
\documentclass[12pt, a5paper, oneside]{book}

\usepackage[utf8]{inputenc}
\usepackage{textcomp}
\usepackage[french]{babel}
\usepackage[T1]{fontenc}
\usepackage{lmodern}

% I import graphicx because there is an image on my title page
\usepackage{graphicx}

% I import hyperref because I use the "href" command in some included files. I use href in these files because I use the same to build my e-books. 
% The hidelinks option prevents to add a style that highlight clickable items. This kind of style could screw up the render on paper. 
\usepackage[pdftex,pdfauthor={Marine Dunstetter},pdftitle={Suzuha},hidelinks]{hyperref}

% I use sectsty package for lazzy customization of the chapter titles
% chapternumberfont is for the chapter number line ("Chapitre 1")
% chaptertitlefont is for the title itself ("La petite danseuse")
\usepackage{sectsty}
\chapternumberfont{\large}
\chaptertitlefont{\Large}

% pagestyle defines the global look of all pages. When using the document class "book", it default to something that actually doesn't fit novels so much. 
% plain is what we want, there is no headers and the page numbers are centered at the bottom of the page.
\pagestyle{plain}

% Default "book" margins let a lot of space at page top and bottom. Also, they dont let the same side margins on even and odd pages because they care about the binding. This is nice, but maybe somehow overkill as printers often advice to let 2cm on each side.
% I use the package geometry to configure the margins I want.
% It can be done without geometry, but in this case you'll have to use commands to add or remove a size from the default margin. And as they are not the same on left and right and it's hard to make sure what are the initial values, I found the use of geometry safer. 
\usepackage[hmargin=2.0cm,vmargin=2.0cm]{geometry}

% A couple of data that can be use as variables elsewhere.
% For instance "\@title" will show the title of the book.
\title{Suzuha}
\author{Marine Dunstetter}
\date{2017}

% \makeatletter ... \makeatother can be required to override some commands.
\makeatletter
% This is to give a better look to table of contents, with a smaller font and dots.
\renewcommand\l@chapter{\@dottedtocline{0}{0em}{1.5em}}
% This is a custom command to define a font huger than huge. I use it for the book title.
\newcommand\HUGE{\@setfontsize\HUGE{35}{50}}
% I entirely rewrite maketitle so the title page display as I want.
\renewcommand{\maketitle}{%
  \begin{titlepage}
    % "center" is to center the content horizontally
    \begin{center}
      \@author \\
      % What \hspace{0pt} + \vfill embeds is centered vertically
      \hspace{0pt}
      \vfill
        \includegraphics[width=132px]{logo-academie-des-renards-black.png}
        \leavevmode \\
        \leavevmode \\
        \leavevmode \\
        {\HUGE\textsc{\@title}}
      \vfill
      \hspace{0pt}
    \end{center}
  \end{titlepage}
}
\makeatother

% My custom \parbr command add more space between two paragraphs. I use it in the text to highligh ellipsis or understandable context change.
\newcommand{\parbr}{%
  \par
  \leavevmode \\
  \par}
  
% My custom command \empty page adds an actual empty white page. This is require at least for the first page of the book, as the printer won't decide for you that nothing should be written on the first page.
% \mbox{} is the important instruction here, because \newpage only makes the next content starts on a new page, it is just a breaker between 2 contents, it does not add empty content.  
\newcommand{\emptypage}{
  \newpage
  \thispagestyle{empty}
  \mbox{}
}

\begin{document}

\emptypage

\newpage
% This will hide the page number on editor's data & legal stuff page.
\thispagestyle{empty}

% This command draws a square on the page, from top to bottom. As result, when you write text, it is "pushed" below the invisible square. It's a bit patch-up, you need to manually tweack the size of the square (20 here) according to the size of your text, but it does the job.
\noindent{\parbox[b][20\baselineskip]{\linewidth}{\rule{0pt}{\linewidth}}}

% footnotesize is one of the default font size. It writes small characters as for footer notes. I thought it was just fine for the editors's data. 
\begin{footnotesize}  

% I remove line identation for these data
\noindent{Édité par Marine Dunstetter, janvier 2022}

\noindent{10120, Saint-André-Les-Vergers, France} \\

\noindent{Première publication en juin 2017 au format e-book} \\

\noindent{Droits réservés} \\

\noindent{Dépôt légal janvier 2022}

\noindent{\textbf{loi n°49-956 du 16 juillet 1949 sur les publications destinées à la jeunesse, modifiée par la loi n°2011-525 du 17 mai 2011 : février 2020}} \\

% I add more space before the ISBN
\leavevmode \\

\noindent{ISBN: 978-2-9561497-6-7}
 
\end{footnotesize}

\maketitle

\emptypage

% This syntaxe is here only to hide the chapter number for "Prologue", it doesn't follow the general diplay rule and it should not be counted as chapter 1. 
\chapter*{%
Prologue\\
\leavevmode \\
Le monstre noir}
\addcontentsline{toc}{chapter}{Prologue}

% Your book should start on an odd page. In my case, it is page 5. I have to reset the current page number here because the white pages created with \emptypage are not automatically counted.
\setcounter{page}{5}

Melle se sent ballottée dans le sac porte-bébé fixé sur le dos de son père. Il redescend la pente d'un pas vif, en faisant tout de même attention à ne pas glisser. Melle est déçue et angoissée. Son père l'emmenait pique-niquer dans leur clairière préférée, plus haut sur la montagne, et il semblait de bonne humeur. Mais lorsqu'ils étaient enfin arrivés, il avait cessé de sourire et avait même décidé de faire demi-tour. Melle ne comprend pas ce qu'elle a fait de mal pour qu'il soit ainsi fâché.

Puis soudain, un hurlement déchirant se fait entendre dans les hauteurs. Son père s'arrête brusquement et se retourne. Il s'éloigne du chemin forestier afin de sortir du couvert des arbres et avoir en vue le haut de la montagne. Melle ne sait pas ce qu'il distingue là-haut, mais il pousse une exclamation de stupeur. Alors qu'il revient vers le chemin en courant, Melle aperçoit brièvement un terrifiant monstre noir. Un lézard géant, noir comme la nuit, dont le corps est parcouru d'éclairs mauves.

\parbr

Le monstre était sorti de la montagne même. À l'endroit où l'eau souterraine débouchait sur une grande cascade, le monstre noir, une créature gigantesque à mi-chemin entre une salamandre et un caméléon, avait déchiré la paroi rocheuse, puis avait commencé à descendre le flanc de la montagne. Les déplacements de roches qu'il avait provoqué avaient considérablement augmenté le débit, et la rivière en aval avait rapidement débordé de son lit. Or cette rivière traversait la ville en contrebas avant de rejoindre l'océan.

Les guerriers-magiciens de l'île s'étaient dépêchés de venir à la rencontre du monstre pour l'arrêter. Constatant la terrible inondation qui se préparait, ceux qui maîtrisaient la magie terrestre avaient fait de leur mieux pour préparer au plus vite un moyen de dévier l'eau. Ils avaient déplacé des roches massives afin de faire barrage au torrent et lui faire emprunter un autre chemin. Leur travail avait été rendu fastidieux par l'approche de la créature noire, qui avait écrasé une partie de la construction avant d'être finalement vaincue et de disparaître dans une myriade d'éclairs mauves.

L'inondation avait fait quelques dégâts, mais le pire avait été évité. Suite à cet événement, le barrage avait été consolidé et repensé de sorte à ce qu'une partie de l'eau traverse la ville comme elle l'avait toujours fait, et que le reste passe par un chenal nouvellement creusé.

\thispagestyle{empty}

\chapter{La petite danseuse}

La fillette pose le bout de son pied sur le plancher. L'autre jambe se lève avec grâce. Elle pivote, lentement, concentrée. Dans chacune de ses petites mains, une pierre bleue et lisse scintille à la lueur des lampes orangées fixées au-dessus de l'estrade. Autour de la fillette sont disposés quelques petits tabourets supportant des bassines d'eau aux trois quarts pleines.

L'eau des bassines, mystérieusement attirée par les pierres bleues, ondule comme un serpent, en suspension dans l'air. Lorsque la fillette tourne sur elle-même, l'eau tourne avec elle, comme un ruban, et lorsqu'elle se déplace, l'eau suit son mouvement. Pas suffisamment vite ni avec suffisamment de légèreté. Ses gestes sont encore trop mécaniques et focalisés sur la maîtrise de l'eau plus que sur la danse. La petite a bien conscience que le spectacle qu'elle offre est encore loin de l'élégance attendue.

Lorsqu'elle dirige son regard vers les sièges du théâtre, elle croise l'expression contrariée de sa préceptrice. Le visage au nez crochu, auquel les yeux maquillés de vert donnent un air de perroquet, est tordu d'une moue désapprobatrice. La fillette relâche une fraction de seconde sa concentration. Ce court instant d'inattention est suffisant. Les rubans d'eau viennent s'éclater sur le plancher.

La fillette est trempée.

La préceptrice soupire. Elle frappe une fois dans ses mains. C'est le signe qu'il est temps d'arrêter l'exercice et qu'il faut maintenant se rendre au réfectoire.

La fillette se dépêche d'aller vider les bassines, épaulée par l'une de ses camarades, tandis que deux autres se chargent de ranger les tabourets. Une cinquième s'empresse de passer un coup de serpillière sur l'estrade. Une fois le petit théâtre arrangé, elles sortent calmement en éteignant les éclairages derrière elles.

Il n'y a jamais beaucoup de soleil sur l'île aux voix, elle est constamment traversée par de gros nuages gris. La fillette s'imagine parfois aux côtés de ces nuages qui, en route vers la destination indiquée par le vent, aperçoivent cet étrange pic qui dépasse de l'océan et, curieux, font un détour pour l'observer de plus près, tournent autour, puis reprennent leur chemin. Néanmoins, une fois dehors, la fillette doit se couvrir les yeux pour s'habituer à la lumière, car le théâtre où les enfants s'exercent ne possède aucune fenêtre, et l'éclairage y est très sobre.

---~Melle~! appelle une voix.

Réomus, adossé au mur du théâtre, sautille les quelques pas qui le séparent de son amie. Elle lui sourit et il lui attrape les mains.

---~Tu danses de mieux en mieux~! s'exclame-t-il, enthousiaste. Ta préceptrice est trop sévère avec toi, elle devrait pourtant admettre que ton niveau est très supérieur à celui des autres filles.

Melle n'est pas d'accord. Mme Lune est sa préceptrice depuis deux ans, lorsqu'elle est entrée à l'école de magie artistique. Elle est très sévère, c'est vrai, mais elle pousse juste ses filles à se dépasser et à être les meilleures. Elle tire sa satisfaction des applaudissements réservés à ses élèves, et qui montrent qu'elle a accompli son devoir avec brio. Il est normal qu'elle soit aussi dure avec chacune d'entre elles. Il est vrai que Melle ne l'aimait pas beaucoup au début, et elle lui fait définitivement penser à un perroquet.

La fillette fait un geste apaisant de la tête pour montrer à Réomus que tout va bien, qu'elle ne s'offusque pas de la façon d'être de Mme Lune.

---~De toute façon, que tu sois douée ou pas ne change rien à mes yeux, renchérit Réomus. Plus tard, tu seras ma danseuse.

Melle acquiesce avec un petit pincement au cœur sur lequel, du haut de ses huit ans, elle a encore du mal à mettre des mots. Réomus est son ami depuis toujours, ils sont inséparables depuis tout petits. Pouvoir compter sur lui est très rassurant, car Melle sait qu'elle obtiendra un travail de danseuse qu'elle réussisse ou qu'elle échoue à l'école. Rassurant aussi, car elle sait que Réomus deviendra une grande personne respectable, qui traitera très bien ses serviteurs, alors que sans lui, elle pourrait tout aussi bien se retrouver dans un foyer où la vie serait difficile.

Malgré tout, au fond d'elle-même, elle ressent quelque chose d'autre, qui ressemble un peu à de la tristesse. C'est sur ce sentiment qu'elle a du mal à mettre des mots. Réomus lui offre un chemin tout tracé sur lequel elle n'a rien à prouver à personne. Mme Lune lui a pourtant donné le goût du travail et du défi. Mais il n'y a pas que cela.

---~Il faut que j'y aille, dit Réomus. À plus tard, Melle~!

Melle le regarde quitter la cour en courant. Elle sent qu'il n'y a pas que cela. Il y a aussi de la jalousie, et un peu de colère. Quelque chose s'est brisé, entre eux. Lui ne semble pas s'en rendre compte, car Melle n'a finalement pas beaucoup changé depuis l'époque où ils passaient tout leur temps ensemble. Mais Réomus, en revanche, n'est plus le même enfant. Lorsqu'ils étaient très jeunes, ils faisaient les mêmes choses, jouaient aux mêmes jeux, et Réomus trouvait cela normal. Mais un an plus tôt, il avait obtenu sa voix.

Et depuis qu'il parle, il dit des choses que Melle n'aime pas. Des choses toutes simples, dont la fillette ne s'est jamais offusquée sur l'instant, mais qui ont agi comme un poison lent.

\parbr

Le théâtre se situe à deux rues de la résidence où loge Melle. Elle s'élance sur le trottoir recouvert d'un marbre gris clair moucheté de gris plus foncé, prend à gauche et traverse la route couverte de gros pavés gris arrondis, continue sur la rue d'en face et longe les façades grises des bâtiments.

Enfin, elle franchit le haut portail métallique. Lui est de couleur bleu nuit, car la magie est couleur, et dans une aile de cette résidence se situe l'école des arts magiques. Une autre aile est utilisée comme dortoir pour les filles qui étudient dans cette école. Melle retourne rapidement à sa chambre pour enfiler des vêtements secs. Elle pense d'abord laisser sécher sa robe sur le rebord de la fenêtre, mais remarque que le temps s'est éclairci.

«~Éclairci~» est un terme tout à fait relatif sur l'île éternellement nuageuse, mais du point de vue de Melle, le soleil s'est maintenant fait suffisamment de place pour lui donner envie de faire un saut à la plage avant la reprise des leçons. Elle attache ses cheveux à l'aveugle, emporte ses vêtements mouillés, grignote rapidement une assiette de pommes de terre au réfectoire et file vers la sortie de la ville.

L'île aux voix est une haute montagne entourée par la mer. La seule ville de l'île est construite sur son flanc. Pour se rendre jusqu'à la côte, Melle descend les rues grises en courant, et la pente douce lui donne l'impression de filer aussi vite que les oiseaux marins. En quelques minutes, le sol pavé devient une route de galets, puis les galets se fondent dans un sable terne un peu caillouteux.

Melle enlève ses sandales et plonge dans le sable. Les coquillages brisés et les algues sèches lui picotent la plante des pieds. Melle aime beaucoup cette sensation. Elle a toujours aimé aller à la plage. En été, il y a de nombreux baigneurs, mais par une journée d'automne comme celle-ci, qui plus est à l'heure où les gens sont encore en train de manger, elle est complètement seule. Elle trouve la mer apaisante.

Tout au bout de la plage, à l'ouest, elle a vue sur une falaise tranchante. Les vagues viennent s'écraser bruyamment contre les rochers. Lorsqu'elle se sent triste ou en colère, Melle court jusqu'à cet endroit et plonge les mains dans l'eau. Elle a alors l'impression que les vagues au loin sont le prolongement de ses mains, et elle se défoule ainsi mentalement contre la falaise.

Mais aujourd'hui, Melle se sent bien. Elle pose ses vêtements mouillés sur un rocher et va tremper ses mollets dans l'eau. Elle sort de ses poches les petites pierres bleues polies et commence à jouer avec l'eau.

C'est son affinité avec la magie aquatique qui lui a permis d'être orientée vers l'école de magie artistique. Sur l'île aux voix, la magie est beauté. Tout le monde aime les spectacles de belle magie. Si Melle n'avait pas été aussi douée, elle aurait pu être orientée vers une école où l'on enseigne des tâches plus classiques comme la cuisine, le tissage ou l'hôtellerie. Toute tâche mérite que quelqu'un s'y attelle, mais sa situation lui semble préférable, car Melle aime beaucoup pratiquer sa magie de l'eau. Malgré tout, elle ne se sent pas pleinement satisfaite.

Elle travaille à nouveau l'exercice des rubans d'eau du théâtre. Elle sait déjà faire beaucoup de choses, pour son âge, elle est déjà capable de danser et de maîtriser l'eau en même temps. Mais certains mouvements lui paraissent plus difficiles que d'autres. Plus elle donne une forme complexe au liquide, plus elle a de difficulté à garder son contrôle. Elle maîtrise parfaitement les rubans d'eau lorsqu'elle reste très proche de la bassine et qu'elle n'en sort pas complètement le ruban, mais se déplacer avec un ruban complètement coupé de sa bassine d'origine et le faire tourner autour d'elle est un exercice bien plus avancé.

Quand elle en a assez, elle pratique son entraînement secret à elle, un exercice qu'elle a hâte d'entreprendre chaque fois qu'elle vient ici, et qu'elle ne fait qu'ici. Elle s'avance dans l'eau jusqu'aux genoux, puis demande à l'eau de s'éloigner. Elle écarte les bras, paumes vers l'eau, et tourne sur elle-même, lentement, pour exprimer «~son espace~». Elle demande à l'eau de ne pas entrer dans son espace. L'eau obéit, s'éloigne. Melle avance davantage, et l'eau s'adapte à son déplacement, comme un aimant repoussé par un autre. Lorsque l'eau autour d'elle est au niveau de son menton, Melle ressent une petite appréhension. C'est lorsqu'elle a atteint ce stade critique que ses progrès ont commencé à se faire plus lents~: ce stade où, si elle perd le contrôle, l'eau fondra sur elle et lui fera boire la tasse, le courant pouvant l'entraîner quelque part où elle n'a pas pied.

À chaque fois, Melle essaie de faire le petit pas de plus, elle se force à aller toujours plus loin. Un pas, encore un dernier petit pas. L'eau est à peu près au niveau de ses yeux. Elle reste bien concentrée et recule vers la plage, jusqu'à ce que l'eau atteigne à nouveau ses genoux. Alors elle relâche tout, et sent l'eau envelopper ses jambes.

La prochaine fois, elle fera mieux. Et un jour, elle sera capable de marcher sous l'eau. En attendant, elle reprend les exercices de son programme scolaire.

Alors qu'elle fait voltiger les vagues et leur donne des allures de serpent, elle aperçoit au loin une forme grise. La forme s'agrandit à mesure qu'elle approche de l'île, et Melle reconnaît bientôt la silhouette d'un paquebot de petite taille. Intriguée, elle reste longtemps les jambes dans l'eau, à le regarder se diriger vers la côte. Le vent marin finit par lui donner froid, et elle va se mettre à l'abri près du rocher où sèche toujours sa robe.

À mesure que le paquebot se rapproche, Melle distingue plus de détails. Il est blanc, une cabine tout en longueur est construite sur l'avant, il y a des hublots sur le côté et il semble être fait de bois. Finalement, il est plutôt grand, en tout cas plus grand que les bateaux de pêche qui sont amarrés plus loin, dans le petit port tout à l'est de la plage. C'est justement vers ce port que se dirige le paquebot.

Melle sait que ce bateau ne vient pas de l'île aux voix, il vient donc du continent. Cela arrive, parfois. Des bateaux viennent et puis repartent. En été, quelques étrangers séjournent parfois sur leur île, mais ils ne sont pas très nombreux, la plage est occupée surtout par les insulaires eux-mêmes. En dehors de la saison chaude, les visites sont encore plus rares. Réomus lui a expliqué que des hommes venaient parfois jusqu'ici pour acheter des pierres magiques. Peut-être n'en ont-ils pas chez eux.

Melle perçoit des mouvements sur le pont, des silhouettes. Elle décide de marcher vers le port de pêche pour mieux voir. Elle emporte sa robe avec elle, car il sera bientôt temps de retourner en ville. Alors qu'elle longe la plage, la fillette remarque un groupe important qui descend de la route principale vers le port. Il y a un homme haut et gras vêtu d'une longue robe dorée qu'elle reconnaît comme étant le Grand Précepteur. Il est suivi par ses serviteurs. Il est accompagné du chef de l'armée et leur procession est encadrée de chaque côté par une rangée de soldats qui marchent en rang. Les guerriers de l'île aux voix... Melle retourne vers l'intérieur des terres afin de rester discrète, puis elle accélère un peu le pas afin de les voir de plus près.

Du côté du bateau, une rampe a été abaissée jusqu'à la langue rocheuse qui sert de quai, et des personnes descendent selon le même esprit géométrique~: des gens bien habillés à l'air très important sont encadrés par des hommes portant des casques et des tuniques pourpres.

Les deux groupes se rencontrent bientôt. Des saluts sont échangés. Melle reste immobile, sans bouger, à bonne distance, mais suffisamment proche pour distinguer les silhouettes assez précisément. Certaines retiennent particulièrement son attention.

Les premières sont évidemment celles des soldats de l'île aux voix. Melle les a toujours admirés. Chaque année, ils paradent dans la ville, font des démonstrations avec leurs bâtons, organisent des duels amicaux durant lesquels ils présentent leurs techniques en spectacle. Elle adore assister à leur défilé. Son propre père fait également partie des guerriers de l'île. Quatre ans plus tôt, Melle les avait vus combattre le grand monstre noir qui avait bien failli noyer leur ville. Ses souvenirs ne sont pas très clairs. Elle se rappelle le monstre au sommet de la montagne, elle se rappelle aussi les guerriers qui l'attaquaient avec leurs lances. Elle se rappelle également de son père, et d'un jeune soldat qui l'avait sauvée du champ de bataille et l'avait conduite en sécurité. Ces guerriers sont des héros aux yeux de Melle. Elle aussi aimerait savoir faire tout ce qu'ils font. Malheureusement, l'entraînement de soldat est réservé aux garçons. Ils doivent à la fois être doués dans les activités physiques et magiques.

Les secondes silhouettes qui intéressent Melle sont celles des soldats étrangers. Leur tenue est plus colorée que celle des soldats de son île, et elle se demande quelles autres similitudes et différences il y a entre eux. Manient-ils aussi la lance et le bâton~? Pratiquent-ils la magie~? Donnent-ils des défilés devant le peuple qu'ils protègent~?

Enfin, la dernière silhouette est une silhouette haute et svelte, qui se tient à gauche du petit personnage qu'elle identifie comme étant le chef des étrangers. Elle a de longs cheveux noirs coiffés en chignon et tient un grand carnet dans une main et un crayon décoré d'une plume dans l'autre. C'est une femme. Melle est surprise de voir une femme au milieu de ce groupe d'inconnus à l'air si important. Sur l'île aux voix, les femmes ne tiennent pas ce genre de rôle. Comme si l'étrangère se sentait observée, elle tourne la tête vers la broussaille au-delà de la plage et voit la fillette. À cette distance, Melle ne distingue pas son expression. La femme continue de l'observer un instant puis dirige à nouveau son regard sur la procession de l'île aux voix.

Tout le groupe se met alors en marche vers la ville.

Melle risque d'être en retard à ses leçons de l'après-midi. Elle contourne la procession par la route qu'elle a empruntée à l'aller et court vers l'amas de constructions tordues qui la dominent, plus haut sur le flanc de montagne. Le retour est toujours plus long et plus fatiguant, car cette fois il faut remonter toute la pente.

\parbr

Cet après-midi-là, Melle a d'abord un cours de langue commune durant lequel elle apprend la langue majoritairement répandue sur le continent. C'est un cours durant lequel elle mémorise des phrases et les écrit sur des parchemins. Elle ne comprend pas toujours très bien en quoi cet apprentissage est utile, car jamais une personne du continent ne lui a adressé la parole. De plus, on raconte que les gens du continent ne comprennent pas les signes, donc même si elle devait répondre à quelqu'un, il ne saurait pas interpréter son attitude ni sa gestuelle.

Cependant, toutes les filles qui apprennent des métiers pouvant amener à se trouver un jour en présence d'étrangers, comme les arts et spectacles et l'hôtellerie, apprennent la langue commune. Le peuple de l'île aux voix est doué dans la mémorisation des langues. C'est peut-être parce qu'ils ne possèdent pas de voix à la naissance que les habitants de l'île comprennent mieux que quiconque à quel point la voix est un don précieux.

À l'âge de sept ans, comme tous les garçons de son âge, Réomus était parti pour la grotte aux souffles, qui se situe plus haut sur la montagne. Melle ne s'est jamais aventurée jusque là. Selon Réomus, c'est un véritable labyrinthe dans lequel soufflent les voix du futur qui tentent de nous avertir des erreurs que nous ne devons pas commettre. Il ne lui avait pas expliqué de quelle façon les garçons obtenaient concrètement une voix dans cette grotte. Melle avait parfois demandé pourquoi seuls les garçons y allaient. La réponse était assez simple~: «~Parce que c'est dans l'ordre des choses. Les femmes n'ont pas besoin de voix pour effectuer les tâches auxquelles elles sont affectées.~»

C'est aussi à l'âge de sept ans, six lorsqu'ils sont de début d'année, que les enfants savent à quelles tâches ils vont être formés à l'avenir. On leur donne un questionnaire qu'ils doivent remplir. Melle avait noté qu'elle aimait la magie. Elle aurait aimé apprendre à combattre au bâton, comme les soldats de l'île qu'elle aime tant. Mais il n'y avait aucune référence au combat dans le questionnaire. Alors elle s'était rabattue sur le spectacle, car c'était ce qui se rapprochait le plus des activités physiques qu'elle aurait voulu faire. Ses talents de magicienne lui avaient permis d'être orientée vers l'école de magie artistique, plus ou moins conformément à son souhait.

Après le cours de langue, la classe doit s'entraîner à la magie. Dans l'enceinte de l'école de magie artistique, il y a une grande cour au sol de roche polie, comme un grand cratère creusé par la nature, dans lequel a été aménagé de quoi permettre aux enfants de s'exercer. Melle travaille dans un bassin rocheux large comme six fois son lit. Il est rempli d'une eau qui lui monte jusqu'au-dessus des chevilles. Ici, Melle apprend à maîtriser l'eau à l'aide d'orbes.

Melle ne peut utiliser que des orbes pour pratiquer la magie, et pas de tome. Seuls les hommes peuvent utiliser les tomes, car il faut une voix pour lire les formules. Le père de Réomus est un grand magicien. Sa bibliothèque est remplie de livres, dont de nombreux tomes de magie. Chez Réomus, il y a une buanderie dont une fenêtre bien cachée est toujours ouverte. On peut s'y glisser, monter au rez-de-chaussée, vérifier qu'on n'entend pas de bruit, puis se dépêcher de traverser le couloir pour courir à l'étage. La bibliothèque est alors la première porte en face. Melle s'y rend de temps en temps, lorsque Réomus lui dit que son père rentrera probablement très tard ce jour-là.

Elle lit alors des tomes contenant des sorts aquatiques. Elle ne peut bien sûr pas les essayer, mais elle mémorise les formules et tente de les prononcer mentalement, comme les prononcerait un homme avec une voix. Réomus ne comprend pas pourquoi ces livres l'intéressent tant, alors qu'ils ne lui sont d'aucune utilité, mais il la laisse les emprunter juste pour lui faire plaisir.

Pour l'heure, Melle maîtrise de son mieux l'eau de son petit cratère. Elle crée des vagues à volonté, qui déforment le reflet des nuages gris, elle les envoie dans la direction qu'elle souhaite, puis elle tourne sur elle-même, formant un tourbillon sur tout le diamètre du cratère. L'eau accompagne tous ses mouvements. Les choses se compliquent un peu lorsqu'elle doit jouer contre les lois de la nature. «~L'eau doit t'obéir avant d'obéir à la gravité~» lui a un jour expliqué Mme Lune. La gravité, c'est quand on lâche un objet et qu'il tombe sur le sol. Quand on renverse de l'eau, elle tombe sur le sol aussi. Au début de son apprentissage, Melle ressentait parfois des difficultés à soulever l'eau et à la faire voltiger selon ses propres règles. Aujourd'hui elle est bien meilleure, mais elle s'impose des exercices plus avancés qui la mettent en difficulté.

Elle sent qu'elle progresse un peu chaque jour. Elle est confiante quant à ses capacités de magicienne. Mais pour être une danseuse-magicienne, il ne suffit pas de faire voler l'eau, il faut aussi danser avec. Melle a tout appris de la danse à l'école, et cette discipline lui a plu. Tout le monde lui dit qu'elle est douée, et cela lui fait plaisir. Il lui reste encore beaucoup de chemin à parcourir pour effectuer des numéros de danse aquatique aussi gracieux et complexes que les danseuses adultes, mais elle sait qu'elle peut y arriver avec de l'entraînement et de la discipline.

\thispagestyle{empty}

\chapter{La femme qui avait une voix}

Alors que Melle poursuit son entraînement, elle aperçoit Mme Lune qui trottine de groupe en groupe. Elle agite les bras comme une volaille trop lourde pour voler. Ses signes semblent évoquer un spectacle, et elle pointe son doigt en direction du bâtiment de l'école.

\parbr

Pour lire la suite, téléchargez le livre complet ;) 

% The text in chapter will appear on the chapter page, but this one is too long to be in the table of contents, so I use "*" to remove it.
\chapter*{Vous aussi, vous maîtrisez la magie~!}
% Then I manually add my own entry in the table of contents for the chapter, but I rename it to something shorter. 
\addcontentsline{toc}{chapter}{Merci !}

% textit is a command for font style italic 
\textit{Merci à toi, lecteur, d'avoir pris un moment pour voyager avec Suzuha. J'espère que cette aventure t'a plu et que tu repasseras un de ces jours par l'Académie des Renards, car elle a fort besoin de toi.} 

\parbr

L'Académie des Renards est une petite collection de romans indépendante. Vous seul avez le pouvoir de la faire connaître en la partageant autour de vous et en la suivant sur les réseaux sociaux. Les Renards comptent sur votre soutien~!

\parbr

% href command to create links in the PDF is provided by hyperref package. I use the same file for e-books and paper book, that's why links are implemented. In "papier.tex", hyperref is configured so no style appears on links and screw up the paper render.
Site web~: \href{http://academie-des-renards.dunstetter.fr}{academie-des-renards.dunstetter.fr} \\

% The \\ at the end of the line is a line breaker that add a little space before the next line. 
Facebook~: \href{https://www.facebook.com/academiedesrenards/}{www.facebook.com/academiedesrenards/} \\

Twitter~: \href{https://twitter.com/academierenards}{@academierenards} \\

Instagram~:  \href{https://www.instagram.com/academiedesrenards/}{@academiedesrenards} \\

Boutique~: \href{https://utip.io/academiedesrenards/shop}{https://utip.io/academiedesrenards/shop}

% After the last links, I jump to a new page for the last acknowledgements, for esthetic purpose.
\newpage

% textbf is a command for font style bold 
\textbf{Quelques mentions spéciales}

\parbr

Merci Guillaume pour ton soutien sans faille et, d'un point de vue plus pragmatique, pour ton coup de main sur la partie technique. J'aurais sans aucun doute ramé bien plus longtemps pour faire aboutir ce projet si tu n'avais pas été là pour me décharger sur la mise en forme. Et j'apprécie vraiment ce week-end entier passé à me relire, toi qui ne lis pourtant jamais. \\

Merci m'man, tu es sans conteste mon filet à fautes le plus efficace, et tes retours cruels et sans pitié m'ont permis d'améliorer mon écrit (mais non, je plaisante, ce n'était pas si méchant). \\

Merci J-S pour la relecture, j'ai trouvé ça vraiment sympa que tu dévores ce livre aussi vite après l'avoir commencé.





% I customize the table of contents title, which is "Table des matières" by default. The shorter version "Table" is often used in novels.
\renewcommand\contentsname{Table}
\tableofcontents

% I add one last page at the end of the book so the printer can add their information.
\emptypage

\end{document}
